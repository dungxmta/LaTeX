\documentclass[12pt,a4paper]{book}
\usepackage[utf8]{vietnam}
\usepackage{amsmath,amsxtra,amssymb,latexsym,amscd,amsthm}
\usepackage[top=3.5cm, bottom=3.0cm, left=3.5cm, right=2cm] {geometry}

\begin{document}
\begin{center}
\section*{\normalsize $\large\boldsymbol{\S 2. }$ CÁC KHÁI NIỆM CƠ BẢN, CÁC TÍNH CHẤT ĐƠN GIẢN}
\end{center}
\fontsize{14pt}{14pt}\selectfont%--cỡ chữ 14pt và khoảng cách dòng 14pt
%\subsection*{2.1. Các khái niệm cơ bản}
\textbf{2.1. Các khái niệm cơ bản\\}
	\par Cho tới nay ta mới chỉ có khái niệm tổng của một dãy hữu hạn các số  (thực). Bây giờ ta mở rộng sang ``tổng của một dãy số vô hạn các số (thực)''.

	Cho một dãy số thực
\[\left\{ {{{\rm{a}}_n}} \right\} = {{\rm{a}}_1},{{\rm{a}}_2},...,{{\rm{a}}_n},...\]\par
	Lập các tổng của $n$ số hạng đầu:\\
\[{A_n} = {{\rm{a}}_1} + {{\rm{a}}_2} + ... + {{\rm{a}}_n}\quad(A_n)\]
ta được một dãy:
\[\left\{ {{A_n}} \right\} = {A_1},{A_2},...,{A_n},...\]
\par\textbf{Định nghĩa}
\begin{itemize}
\item[--] Ta gọi $A_n$ là \textit{tổng riêng} (thứ $n$) của \textit{chuỗi số} (gọi tắt là chuỗi)
\[\sum\limits_{n = 1}^\infty  {{{\rm{a}}_n}}  = {{\rm{a}}_1} + {{\rm{a}}_2} + ... + {{\rm{a}}_n} + ... \quad (A)\]
gọi là \textit{số hạng tổng quát} của chuỗi
\item[--] Nếu dãy ${A_n}$ có giới hạn ( hữu hạn):
\[A = \mathop {\lim }\limits_{n \to \infty } {A_n}\]
\end{itemize}
thì ta nói $A$ là tổng của chuỗi $(A)$ và viết:
\[{A = }\sum\limits_{n = 1}^\infty  {{{\rm{a}}_n}}  = {{\rm{a}}_1} + {{\rm{a}}_2} + ... + {{\rm{a}}_n} + ...\]

Khi đó ta cũng nói rằng chuỗi $(A)$ hội tụ về $A$.
\begin{itemize}
\item[--] Nếu dãy ${A_n}$ không có giới hạn hữu hạn thì ta nói là chuỗi $(A)$ không hội tụ hoặc \textit{phân kì}.\\

Quan hệ giữa sự hội tụ của chuỗi và của dãy:
\item[--] Như vậy, theo định nghĩa trên, sự hội tụ của chuỗi $(A)$ tương đương với sự hội tụ của dãy tổng riêng $(A_n)$ của nó.
\item[--] Ngược lại, với một dãy bất kì
\[\left\{ {{x_n}} \right\} = {x_1},{x_2},...,{x_n},...\]
\end{itemize}
sự hội tụ của nó có thể quy về sự hội tụ của chuỗi:
\[{x_1} + \sum\limits_{n = 1}^\infty  {({x_n} - {x_{n - 1}})}  = {x_1} + ({x_2} - {x_1}) + ({x_3} - {x_2}) + ... + ({x_n} - {x_{n - 1}}) + ...\]
(chuỗi này có các tổng riêng chính là các số hạng của dãy trên, và thường được gọi là \textit{chuỗi co}).

	Nói một cách khác, việc nghiên cứu sự hội tụ của chuỗi số chỉ đơn giản là một \textit{dạng thức mới} của việc nghiên cứu sự hội tụ của dãy số.
Nhưng, như dưới đây ta sẽ thấy, dạng thức này đem lại lợi ích trong các vấn đề được đặt ra.\\

\par\textbf{Ví dụ 1}

Cho dãy ${{\rm{a}}_n}$ với ${\rm{a}}_n=\rm{a}$ (hằng số) với mọi $n$\par
Khi đó chuỗi
\[\sum\limits_{n = 1}^\infty  {{{\rm{a}}_n}} {\rm{ = }}\sum\limits_{n = 1}^\infty  {\rm{a}}  = {\rm{a}} + {\rm{a}} + ... + {\rm{a + }}...\]
(có các tổng riêng $S_n=n\rm{a}$) hội tụ khi $\rm{a=0}$, phân kì ra $+\infty$ nếu $\rm{a>0}$, ra $-\infty$ nếu $\rm{a<0}$.\\

\par\textbf{Ví dụ 2}

Chuỗi
\[\sum\limits_{n = 1}^\infty  {{{( - 1)}^{n - 1}}}  = 1 - 1 + 1 - 1 + ... + {( - 1)^{n - 1}} + ...\]
có các tổng riêng:
$S_n = \left\{\begin{array}{l}
1 \ \ \ \ \textrm{nếu $n$ lẻ}\\
0 \ \ \ \ \textrm{nếu $n$ chẵn}
\end{array}\right.$\\\\

Do đó $S_n$ không có giới hạn khi $n \to \infty$, tức là chuỗi trên phân kì.\\

\par\textbf{Ví dụ 3}

Nghiên cứu sự hội tụ của chuỗi
\[\sum\limits_{k = 1}^\infty  {\dfrac{1}{{(k + 1)(k + 2)}}}  = \dfrac{1}{{2.3}} + \dfrac{1}{{3.4}} + ... + \dfrac{1}{{(k + 1)(k + 2)}} + ...\]\par
Ta biết rằng
\[\dfrac{1}{{(k + 1)(k + 2)}} = \dfrac{1}{{k + 1}} - \dfrac{1}{{k + 2}}\]\par
Vì thế sự hội tụ của chuỗi trên quy về sự hội tụ của chuỗi co 
\[\sum\limits_{k = 1}^\infty  {\left( {\dfrac{1}{{k + 1}} - \dfrac{1}{{k + 2}}} \right)} \]
có các tổng riêng là:
\[{S_n} = \left( {\dfrac{1}{2} - \dfrac{1}{3}} \right) + \left( {\dfrac{1}{3} - \dfrac{1}{4}} \right) + ... + \left( {\dfrac{1}{{n + 1}} - \dfrac{1}{{n + 2}}} \right)\]
\end{document}
