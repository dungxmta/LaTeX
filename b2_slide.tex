\documentclass[compress,  hyperref={unicode, bookmarks=true, pdfpagemode=FullScreen}]{beamer}
\mode<presentation>
%\usetheme{Madrid}
\useoutertheme[subsection=false]{smoothbars}
\beamertemplateshadingbackground{yellow!0}{white}
\usepackage[utf8]{inputenc}

%\usepackage{utopia} %font utopia imported

\usetheme{Boadilla}
\usecolortheme{wolverine}

\usepackage{amsmath, amssymb}
\usepackage[utf8]{vietnam}
\usepackage{indentfirst}
\usepackage[mathscr]{eucal}
\usepackage{color}
\usepackage{graphicx}
\usepackage{indentfirst}                 
\usepackage{tabvar}
\usepackage{picinpar}
\usepackage{slashbox}                  
\usepackage[unicode]{hyperref}

\newcommand{\xanh}[1]{\textcolor{green}{#1}}
\newcommand{\duong}[1]{\textcolor{blue}{#1}}

\title[Bài giảng]{\bf \Large{{   $\boldsymbol{\S 2. }$ CÁC KHÁI NIỆM CƠ BẢN, CÁC TÍNH CHẤT ĐƠN GIẢN  }}}
\author[N. T. Hà]{\bf \large \textcolor{blue}{Nguyễn Thị Hà}}
\institute[Đại học Thủ đô Hà Nội]{}
\date{Hà Nội 2015}

%\usepackage[timeinterval=1]{tdclock}
\begin{document}
\frame{%\initclock
\transblindshorizontal
\titlepage}
\begin{frame}{Nội dung bài dạy}
  \tableofcontents
\end{frame}
\fontsize{15pt}{19pt}  \selectfont

\section{\bf $\boldsymbol{\S 2. }$ Các khái niệm cơ bản, các tính chất đơn giản}
\subsection{\bf 2.1. Các khái niệm cơ bản}
\subsection{\bf 2.2. Ví dụ}
\begin{frame} {\bf 2.1. Các khái niệm cơ bản}
\pause 
Cho tới nay ta mới chỉ có khái niệm tổng của một dãy hữu hạn các số  (thực). \pause Bây giờ ta mở rộng sang ``tổng của một dãy số vô hạn các số (thực)''.
\end{frame}

%--Trang4
\begin{frame}{2.1. Các khái niệm cơ bản (t)}
Cho một dãy số thực:
\[\left\{ {{a_n}} \right\} = {a_1},{a_2},...,{a_n},...\]
\pause
Lập các tổng của $n$ số hạng đầu:
\[{A_n} = {a_1}+{a_2}+...+{a_n}\quad(A_n)\]
\pause
ta được một dãy:
\[\left\{ {{A_n}} \right\} = {A_1},{A_2},...,{A_n},...\]
\end{frame}

%--Trang5
\begin{frame}{2.1. Các khái niệm cơ bản (t)}
\[\left\{ {{A_n}} \right\} = {A_1},{A_2},...,{A_n},...\]
\setbeamercolor{block title}{use=structure,fg=white,bg=green!75!black}
\setbeamercolor{block body}{use=structure,fg=black,bg=black!3!white}
\begin{block}{\bf Định nghĩa}
\pause
Ta gọi $A_n$ là \duong{\textit{tổng riêng}} (thứ $n$) của \textit{chuỗi số} (gọi tắt là chuỗi)
\pause
\[\sum\limits_{n = 1}^\infty  {{a_n}}  = {a_1}+{a_2}+...+{a_n}+ ... \quad (A)\]
gọi là \duong{\textit{số hạng tổng quát}} của chuỗi.
\end{block} 
\end{frame}

%--Trang6
\begin{frame}
\begin{block}{}
\begin{itemize}
\item[•]
Nếu dãy ${A_n}$ có giới hạn ( hữu hạn):
\[A = \mathop {\lim }\limits_{n \to \infty } {A_n}\]
\pause
thì ta nói $A$ là tổng của chuỗi $(A)$ và viết:
\[{A = }\sum\limits_{n = 1}^\infty  {{a_n}}  = {a_1}+{a_2}+...+{a_n}+ ...\]
\pause
Khi đó ta cũng nói rằng chuỗi $(A)$ \duong{hội tụ} về $A$.\\
\pause
\item[•]
Ngược lại, nếu dãy ${A_n}$ không có giới hạn hữu hạn thì ta nói là chuỗi $(A)$ \duong{không hội tụ} hoặc \duong{phân kì}.
\end{itemize}
\end{block}
\end{frame}

%--Trang7
\begin{frame}
\underline{\duong{\bf Quan hệ giữa sự hội tụ của chuỗi và của dãy:}}
\pause
\begin{itemize}
\item[•] Theo định nghĩa trên, sự hội tụ của chuỗi $(A)$ tương đương với sự hội tụ của dãy tổng riêng $(A_n)$ của nó.
\pause
\item[•] Ngược lại, với một dãy bất kì
\[\left\{ {{x_n}} \right\} = {x_1},{x_2},...,{x_n},...\]
\pause
sự hội tụ của nó có thể quy về sự hội tụ của chuỗi:
\[{x_1} + \sum\limits_{n = 1}^\infty  {({x_n} - {x_{n - 1}})}  = {x_1} + ({x_2} - {x_1}) + ... + ({x_n} - {x_{n - 1}}) + ...\]
\pause
(chuỗi này có các tổng riêng chính là các số hạng của dãy trên, và thường được gọi là \duong{\textit{chuỗi co}}).
\end{itemize}
\end{frame}

%--Trang8
\begin{frame}{2.1. Các khái niệm cơ bản (t)}
\begin{block}{\bf Nhận xét}
Nói một cách khác, việc nghiên cứu sự hội tụ của chuỗi số chỉ đơn giản là một \textit{dạng thức mới} của việc nghiên cứu sự hội tụ của dãy số.\\
\medskip %phân đoạn văn
Nhưng, như dưới đây ta sẽ thấy, dạng thức này đem lại lợi ích trong các vấn đề được đặt ra.
\end{block}
\end{frame}

%--Trang9
\begin{frame}{\bf 2.2. Ví dụ}
\pause
\underline{\bf Ví dụ 1.}
Cho dãy ${a_n}$ với ${a_n}=a$ (hằng số) với mọi $n$.\\
\pause
Khi đó chuỗi
\[\sum\limits_{n = 1}^\infty  {{a_n}} = \sum\limits_{n = 1}^\infty  {a} = a + a + ... + a +  ...\]
(có các tổng riêng $S_n=na$)\\ hội tụ khi $a=0$\pause ,\\ phân kì ra $+\infty$ nếu $a>0$\pause , ra $-\infty$ nếu $a<0$.
\end{frame}

%--Trang10
\begin{frame}
\underline{\bf Ví dụ 2.}
\pause
Chuỗi
\[\sum\limits_{n = 1}^\infty  {{{( - 1)}^{n - 1}}}  = 1 - 1 + 1 - 1 + ... + {( - 1)^{n - 1}} + ...\]
\pause
có các tổng riêng:
$S_n = \left\{\begin{array}{l}
1 \ \ \ \ \textrm{nếu $n$ lẻ}\\
0 \ \ \ \ \textrm{nếu $n$ chẵn}
\end{array}\right.$\\
\medskip
Do đó $S_n$ không có giới hạn khi $n \to \infty$\pause , tức là chuỗi trên phân kì.
\end{frame}
%--Trang11
\begin{frame}
\underline{\bf Ví dụ 3.}
\pause
Nghiên cứu sự hội tụ của chuỗi
\[\sum\limits_{k = 1}^\infty  {\dfrac{1}{{(k + 1)(k + 2)}}}  = \dfrac{1}{{2.3}} + \dfrac{1}{{3.4}} + ... + \dfrac{1}{{(k + 1)(k + 2)}} + ...\]
\pause
Ta biết rằng
\[\dfrac{1}{{(k + 1)(k + 2)}} = \dfrac{1}{{k + 1}} - \dfrac{1}{{k + 2}}\]
\end{frame}
%--Trang12
\begin{frame}
Vì thế sự hội tụ của chuỗi trên quy về sự hội tụ của \textit{chuỗi co} 
\[\sum\limits_{k = 1}^\infty  {\left( {\dfrac{1}{{k + 1}} - \dfrac{1}{{k + 2}}} \right)} \]
\pause
có các tổng riêng là:
\[{S_n} = \left( {\dfrac{1}{2} - \dfrac{1}{3}} \right) + \left( {\dfrac{1}{3} - \dfrac{1}{4}} \right) + ... + \left( {\dfrac{1}{{n + 1}} - \dfrac{1}{{n + 2}}} \right)\]
\end{frame}
\end{document}