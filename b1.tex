\documentclass[12pt]{article}
\usepackage[utf8]{vietnam}
\usepackage{amsmath,amsxtra,amssymb,latexsym,amscd,amsthm}
\usepackage[a4paper,left=1.5cm,right=1.5cm,top=2cm,bottom=1.5cm]{geometry}
\renewcommand{\baselinestretch}{1.4}% khoảng cách giữa các dòng

\begin{document}
\thispagestyle{empty}
\begin{tabular}{c c c c c}
\textbf{BỘ \underline{GD\&ĐT VĨNH P}HÚC}&&&&\textbf{ĐỀ KSCL ÔN THI QUỐC GIA LẦN 1} \\
&&&&\textbf{NĂM HỌC 2014 - 2015} \\
&&&&\textbf{MÔN: TOÁN}\\
&&&&\textit{Thời gian làm\underline{ bài: 180 phút, không kể thời }gian phát đề}\\\\
\end{tabular}

%\vspace*{1cm}
% \noindent: bỏ khoảng trắng đầu dòng
%Câu1---
\noindent\textbf{Câu 1 (4 điểm).}
Cho hàm số $y = \dfrac{{2x - 1}}{{x - 1}} $.\par
a) Khảo sát sự biến thiên và vẽ đồ thị $(C)$ của hàm số đã cho.\par
b) Viết phương trình tiếp tuyến của đồ thị $(C)$ biết tiếp tuyến vuông góc với đường thẳng có phương trình $y = x + 2015$.

%Câu2---
\noindent\textbf{Câu 2 (2 điểm).}
Giải các phương trình sau:\par
a) $2\sin^2x + 3\sin x - 2 = 0$\par
b) $\displaystyle\log _2x + {\log _2}(x - 2) = {\log _2}(6 - x)$

%Câu3---
\noindent\textbf{Câu 3 (2 điểm).}
Tìm giá trị lớn nhất và giá trị nhỏ nhất của hàm số $f(x) = x^3 - 3x + 2$ trên đoạn $[0;2]$.

%Câu4---
\noindent\textbf{Câu 4 (2 điểm).}
Xếp ngẫu nhiên 3 học sinh nam và 2 học sinh nữ thành một hàng ngang. Tính xác suất để có 2 học sinh nữ đứng cạnh nhau.

%Câu5---
\noindent\textbf{Câu 5 (2 điểm).}
Cho hình chóp $S.ABCD$ có đáy $ABCD$ là hình chữ nhật, $AB=a$, $AD=a\sqrt{3}$, $SA\perp(ABCD)$, góc giữa mặt phẳng $(SBD)$ và mặt phẳng $(ABCD)$ bằng $60^\circ$. Tính theo $a$ thể tích khối chóp $S.ABCD$ và khoảng cách giữa hai đường thẳng $AC$ và $SD$.

%Câu6---
\noindent\textbf{Câu 6 (2 điểm).}
Trong mặt phẳng với hệ tọa độ $Oxy$, cho tam giác $ABC$ có trực tâm $H(3;0)$ và trung điểm $BC$ là $I(6,1)$. Đường thẳng $AH$ có phương trình $x + 2y - 3 = 0$. Gọi $D$, $E$ lần lượt là chân đường cao kẻ từ $B$ và $C$ của tam giác $ABC$. Xác định tọa độ các đỉnh tam giác $ABC$, biết đường thẳng $DE$ có phương trình $x - 2 = 0$ và điểm $D$ có tung độ dương.

%Câu7---
\noindent\textbf{Câu 7 (2 điểm).}
Cho hình trụ có hai đáy là hai hình tròn có tâm $O$ và $O'$, bán kính bằng $a$. Hai điểm $A$, $B$ lần lượt nằm trên hai đường tròn tâm $O$ và $O'$ sao cho $AB$ hợp với trục $OO'$ một góc $45^\circ$ và khoảng cách giữa chúng bằng $\dfrac{a\sqrt{2}}{2}$. Tính theo $a$ diện tích toàn phần của hình trụ đã cho.

%Câu8---
\noindent\textbf{Câu 8 (2 điểm).}
Giải hệ phương trình $\left\{\begin{array}{l}
xy + 2 = y\sqrt{x^2+2}\\
y^2 + 2(x + 1)\sqrt{x^2 + 2x + 3} = 2x^2 - 4x
\end{array}\right. \ \ \ \ (x, y \in \mathbb{R})$.

%Câu9---
\noindent\textbf{Câu 9 (2 điểm).}
Cho $x$, $y$, $z$ là các số thực dương thảo mãn $x + y + z = -1$. Tìm giá trị lớn nhất của biểu thức $P = \dfrac{x^3y^3}{(x + yz)(y + xz)(z + xy)^2}$.

\begin{center}{\bf ---------------------Hết---------------------\\
Thí sinh không được sử dụng tài liệu. Cán bộ coi thi không giải thích gì thêm!\\}\end{center}
Họ và tên thí sinh: .....................................................;	Số báo danh: .....................................................
\end{document}